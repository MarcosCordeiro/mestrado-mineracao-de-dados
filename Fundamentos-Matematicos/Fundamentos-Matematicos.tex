\documentclass[a4paper, 12pt]{article}
 
\usepackage[portuges]{babel}
\usepackage[utf8]{inputenc}
\usepackage{amsmath}
\usepackage{indentfirst}
\usepackage{graphicx}
\usepackage{multicol,lipsum}
\usepackage{design_ASC}
\usepackage{enumitem}

\setlength\parindent{0pt}
\title{Fundamentos Matemáticos} 
\author{Marcos Cordeiro de B. Jr\\ 
PPGEEC - Programa de Pós Graduação em Engenharia Elétrica e Computação\\
\textsc{\textbf{Universidade Presbiteriana Mackenzie}}
}
\date{Março de 2020} 


\begin{document}
  
\setlength{\droptitle}{-5em}    
\maketitle

\section*{Questão 1 - Multiplicação de Matrizes}
  {\bfseries Efetue a multiplicação entre as matriz $A = 
  \begin{bmatrix}
    1 & 2 \\
    0 & 1
  \end{bmatrix}$ 
  e $B = 
  \begin{bmatrix}
    4 & -1 \\
    5 & 0
  \end{bmatrix}$ 
  e verifique se a propriedade comutativa é válida. }

  \section*{Resposta}
  Multiplicação das Matrizes $A \cdot B$:
  \begin{center}
    $A \cdot B = 
    \begin{bmatrix}
      1 & 2 \\
      0 & 1
    \end{bmatrix} 
    \cdot
    \begin{bmatrix}
      4 & -1 \\
      5 & 0
    \end{bmatrix}$

    $A \cdot B = 
    \begin{bmatrix}
      1 \cdot 4 + 2 \cdot 5 & 1 \cdot (-1) + 2 \cdot 0 \\
      0 \cdot 4 + 1 \cdot 5 & 0 \cdot (-1) + 1 \cdot 0 
    \end{bmatrix}$

    $A \cdot B = 
      \begin{bmatrix}
        14 & -1 \\
        5 & 0
      \end{bmatrix}$
  \end{center}

Multiplicação das Matrizes $B \cdot A$:
\begin{center}
  $B \cdot A =
    \begin{bmatrix}
      4 & -1\\
      5 & 0
    \end{bmatrix}
    \cdot
    \begin{bmatrix}
      1 & 2 \\
      0 & 1
    \end{bmatrix}$

    $B \cdot A =
    \begin{bmatrix}
      4 \cdot 1 + (-1) \cdot 0 & 4 \cdot 2 + (-1) \cdot 1 \\
      5 \cdot 1 + 0 \cdot 0 & 5 \cdot 2 + 0 \cdot 1
    \end{bmatrix}
    $

    $B \cdot A = 
    \begin{bmatrix}
      4 & 7 \\
      5 & 10 
    \end{bmatrix}
    $
\end{center}

Portanto $ A \cdot B \neq B \cdot A $ e por isso a propriedade comutativa não é válida.
\newpage

\section*{Questão 2 - Determinante de Matrizes}
  {\bfseries Dado uma Matriz 
  $A
  \begin{bmatrix}
    4 & 7\\
    -1 & 5
  \end{bmatrix}$
  , sendo $A^t$ sua transposta, qual o determinante da matriz $A.A^t$? 
  }

  \section*{Resposta}
    Multiplicação das Matrizes: \\
    \begin{center}
      $A.A^t = 
        \begin{bmatrix}
          4 & 7\\
          -1 & 5
        \end{bmatrix} 
        \cdot
        \begin{bmatrix}
          4 & -1 \\
          7 & 5
        \end{bmatrix}
      $
      
      $A.A^t = 
        \begin{bmatrix}
          4 \cdot 4 + 7 \cdot 7 & 4 \cdot (-1) + 7 \cdot 5 \\
          (-1) \cdot 4 + 5 \cdot 7 & (-1) \cdot (-1) + 5 \cdot 5
        \end{bmatrix}
      $

      $A.A^t = 
        \begin{bmatrix}
          65 & 31 \\
          31 & 26
        \end{bmatrix}
      $
    \end{center}
    Buscando a determinante:

    \begin{center}
      $Det(A.A^t)= 65 \cdot 26 - 31 \cdot 31$  \\
      $Det(A.A^t) = 1690 - 961 $ \\
      $Det(A.A^t) = 729$
    \end{center}

\section*{Questão 3 - Produto Interno}
{\bfseries Cálculo o valor do produto interno entre os vetores $ 
\vec{x} = (3,1,4)$ 
e $ \vec{y} = (6,0,2)$.

\section*{Resposta}
\begin{center}
  $\vec{x} \cdot \vec{y} = (3,1,4) \cdot (6,0,2) $ \\
  $\vec{x} \cdot \vec{y} = 3 \cdot 6 + 1 \cdot 0 + 4 \cdot 2$ \\
  $\vec{x} \cdot \vec{y} = 18 + 0 + 8$ \\
  $\vec{x} \cdot \vec{y} = 26$
\end{center}
\newpage
\section*{Question 4 - Probabilidade}
{\bfseries 
  No jogo de RPG (Role Playing Game), os jogadores utilizam diversos dados com várias lados que variam entre 4, 6, 8, 10, 12 e 20 faces. Essa variação no número de lados serve para controlar o grau de aleatoriedade e equilibrar as regras do sistema de jogo.\\
  Em uma partida entre amigos, um jogador precisa matar um monstro que tem 35 pontos de defesa. Esse jogador terá que jogar o dado de 20 lados duas vezes.\\
  Qual a Probabilidade desse jogador tirar mais que 35 pontos e matar o monstro?
  }

  \section*{Resposta}

  Fórmula:
  \newline
  \begin{center}
  $P = \dfrac{\text{soma da possibilidades dos dados}}{\text{quantidade de combinações máximas dos dois dados}}$ 
  $= \dfrac{S}{J}$
  \newline
\end{center}


Possibilidades de vencer:

\begin{itemize}
  \item \textit Se dado 1 tirar 16, dado 2 pode tirar 20;
  \item \textit Se dado 1 tirar 17, dado 2 pode tirar 20 ou 19;
  \item \textit Se dado 1 tirar 18, dado 2 pode tirar 20, 19 ou 18;
  \item \textit Se dado 1 tirar 19, dado 2 pode tirar 20, 19, 18 ou 17;
  \item \textit Se dado 1 tirar 20, dado 2 pode tirar 20, 19, 18, 17 ou 16;
  
\end{itemize}

Soma possibilidades do dado 2.  
\begin{center}
  $S = [1 + 2 + 3 + 4 + 5]$ \\
  $S = 15$
\end{center}
Quantidade máxima de possibilidades dos dois dados.
\begin{center}
  $J = 20 \cdot 20 $ \\
  $J = 400$
\end{center} 
Probabilidades:

\begin{center}
  $P = \dfrac{S}{J}$
  $ = \dfrac{15}{400}$
  $ = \dfrac{3}{80}$
\end{center}
\newpage
\section*{Questão 5 - Probabilidade Condicional}
{\bfseries Em outra rodada da mesma partida de RPG do exercício anterior, o jogador com os dados pode ganhar um item muito valioso para seu personagem. Porém, para que isso ocorra, ele precisa jogar o dado de 6 lados e tirar um número impar ou maior que dois. \\
Quais são as probabilidades do jogador ganhar esse item?}

\section*{Resposta}
Evento $A$ - Tirar o número ímpar.
\newline
\begin{center}
  $P(A) = \dfrac{\text{números ímpares do dado}}{\text{quantidade de faces do dado}}$

  $P(A) = \dfrac{3}{6}$\\
\end{center}

Evento $B$ - Tirar número maior que 2.
\newline
\begin{center}
  $P(B) = \dfrac{\text{quantidade de números possíveis}}{\text{quantidade de faces do dado}}$\\

  $P(B) = \dfrac{4}{6}$\\
\end{center}

Identificando os elementos comuns entre os eventos A e B.

\begin{center}
  $A \cap B = $ \{3,5\}  \\
  $P(A \cap B) = \dfrac{2}{6} $
\end{center}

Cálculo da probabilidade dos eventos pela \textit{lei da adição}.

\begin{center}
  $P(A \cap B) = P(A)+P(B)-P(A \cup B) $\\
  $P(A \cap B) = \dfrac{3}{6} + \dfrac{4}{6} + \dfrac{2}{6} $ \\
  $P(A \cap B) =  \dfrac{5}{6} $
\end{center}
\newpage
\section*{Questão 6 - Teorema de Bayes}
{\bfseries Uma empresa produz 4\% de peças defeituosas. O controle de qualidade da empresa é realizado em duas
etapas independentes. A primeira etapa acusa uma peça defeituosa com 80\% de probabilidade de acerto. A segunda etapa acusa uma peça defeituosa com 90\% de probabilidade.\\
Calcule a probabilidade de que:

\begin{enumerate}[label=\alph*]
  \item Uma peça defeituosa passe pelo controle de qualidade.
  \item Ao adquirir uma peça produzida por esta empresa, ela seja defeituosa.
\end{enumerate}

\section*{Resposta}

Probabilidade de que a peça defeituosa passe pelo controle de qualidade.

\begin{center}
  $P[ ({E_1} | D) \cap ({E_2} | D) ] = 0.20 \cdot 0.10 = 0.02$
\end{center}

Probabilidade de que seja defeituosa a peça adquirida desta empresa:
\begin{center}
  $P(D) = \text{4\% de 2\%} = 0.04 \cdot 0.02 = 0.00008 \text{ ou } 0.08\%$
\end{center}

\end{document}